\documentclass[12pt,a4paper]{article}
\usepackage[T2A]{fontenc}
\usepackage[utf8]{inputenc}
\usepackage[english,russian]{babel}
\usepackage{amssymb}
\usepackage{amsfonts}
\usepackage{amsmath}
\usepackage{cmap}
\usepackage{fancyhdr}
\usepackage{enumitem}
\usepackage{multicol}
\usepackage{url}
\usepackage[center]{titlesec}
\usepackage{mathtools}
\textwidth=17cm
\voffset=-1cm
\hoffset=-0.5cm
\topmargin=0cm
\textheight=24.5cm
\oddsidemargin=0pt

\linespread{1.1}

\pagestyle{fancy}
\lhead{\bfseries Storehouse Automation}
\rhead{\itshape 2019/2020}
\fancyfoot{}

\renewcommand{\theenumii}{\asbuk{enumii}}
\AddEnumerateCounter{\asbuk}{\@asbuk}{\cyrm}

\renewcommand{\to}{\longrightarrow}
\newcommand{\tof}{\to\!\!.\ }
\newcommand{\tod}{\Longrightarrow}
\newcommand{\xtod}[1]{\xRightarrow{\text{#1\ } }}
\newcommand{\alphabet}[2]{#1=\{#2\}}
\newcommand{\salp}[1]{\alphabet{\Sigma}{#1}}

\begin{document}
\section*{Документация к проекту(курсовой работе) Storehouse Atomation.}
\renewcommand{\theenumii}{\asbuk{enumii}}
\AddEnumerateCounter{\asbuk}{\@asbuk}{\cyrm}
\hrulefill

\subsection*{Файл storehouse.py.}
\subsubsection*{Файл содержит описание среды, с которой взаимодействует агент.}

\begin{enumerate}
\item Метод \_init\_(self) является конструктором класса среды. Происходит инициализация атрибутов:
\begin{enumerate}
\item Атрибут action\_space описывает формат допустимых действий.
\item Атрибут observation\_space описывает формат допустимых наблюдений.
\item Атрибут state описывает представление склада в виде матрицы, заполненной нулями.
\end{enumerate}

\end{enumerate}
\end{document}

%%% Local Variables:
%%% mode: latex
%%% TeX-master: t
%%% End:
